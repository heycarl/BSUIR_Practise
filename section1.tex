\section{Понятие интернет вещей (IoT – Internet of Things)}

\subsection{Определение}
IoT (Интернет вещей) — это передовая система автоматизации и аналитики, которая использует
сети, датчики, большие данные и технологии искусственного интеллекта для создания полных систем
для продукта или услуги\cite{IoT}. Эти системы обеспечивают большую прозрачность, контроль и производительность, а так же применимы к любой отрасли или системе.

\subsection{Применения}
Системы IoT находят применение в различных отраслях благодаря своей уникальной гибкости, их назначение может
подходить в любой среде. Эти системы улучшают сбор данных, автоматизацию, операции, используя
интеллектуальные устройства и мощные передовыу технологии.

\subsection{Реализации}
Системы IoT позволяют пользователям добиться более глубокой автоматизации, анализа и интеграции в рамках системы.
Они улучшают охват этих областей и их точность. Интернет вещей использует существующие и новые
технологии для зондирования, создания сетей и робототехники.
Интернет вещей использует последние достижения в области программного обеспечения, способствует падению цен на оборудование и современное отношение к
технологии. Его новые и усовершенствованные элементы вносят серьезные изменения в доставку продуктов,
товарам и услугам в области IOT.