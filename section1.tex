\section{Понятие интернет вещей (IoT – Internet of Things)}

\subsection{Определение}
IoT (Интернет вещей) — это передовая система автоматизации и аналитики, которая использует
сети, датчики, большие данные и технологии искусственного интеллекта для создания полных систем
для продукта или услуги\cite{IoT}. Эти системы обеспечивают большую прозрачность, контроль и производительность.
применительно к любой отрасли или системе.

\subsection{Применения}
Системы IoT находят применение в различных отраслях благодаря своей уникальной гибкости и способности быть
подходит в любой среде. Они улучшают сбор данных, автоматизацию, операции и многое другое.
больше благодаря интеллектуальным устройствам и мощным передовым технологиям.
Это руководство призвано предоставить вам подробное введение в IoT. Он вводит ключ
концепции Интернета вещей, необходимые при использовании и развертывании систем Интернета вещей.

\subsection{Реализации}
Системы IoT позволяют пользователям добиться более глубокой автоматизации, анализа и интеграции в рамках системы.
Они улучшают охват этих областей и их точность. Интернет вещей использует существующие и новые
технологии для зондирования, создания сетей и робототехники.
Интернет вещей использует последние достижения в области программного обеспечения, падение цен на оборудование и современное отношение к
технологии. Его новые и усовершенствованные элементы вносят серьезные изменения в доставку продуктов,
товары и услуги; и социальные, экономические и политические последствия этих изменений.