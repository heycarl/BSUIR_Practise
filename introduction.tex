\csection{Введение}
Учебная практика является обязательным элементом подготовки специалиста с высшим образованием и одной из форм текущей аттестации студента по учебной дисциплине. Для студентов это первая учебная работа такого рода и объёма. Реферат содержит результаты теоретического исследования по теме «Концепция Интернет вещей (IoT)», что способствует получению базовых навыков по поиску и систематизации информации, а также подготовке к дальнейшему изучению технологий, которые относятся к концепции IoT.


В первом разделе дается базовая информация о концепции интернета вещей, а так же возможные области применения. Описываются задачи, которые помогает решать данная область, возможные способы реализации.


Второй раздел описывает основные характеристики технологии IoT, методы использования устройств, которые можно отнести к заданной концепции, а так же части, без которых концепцию невозможно будет реализовать. Третий раздел информирует о преимуществах технологии, а так же о недостатках. 


Четвертый и пятый разделы описывают возможные реализации \\
устройств экосистемы IoT, а так же используемое программное обеспечение, которое заложено в реализацию концепции IoT.


Раздел шесть и семь содержат информацию об используемых протоколах передачи информации между устройствами в сети, а так же более подробно описывают аспекты безопасности системы интернета вещей.


Текст реферата был проверен при помощи системы Антиплагиат. Отчет о проверке приведен в приложении А.