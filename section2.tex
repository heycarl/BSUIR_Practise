\section{Основные характеристики}

Наиболее важные функции IoT включают искусственный интеллект, возможность подключения, датчики, активные
вовлеченность и использование небольших устройств. Краткий обзор этих функций приведен ниже:
\begin{itemize}
    \item ИИ – Интернет вещей делает практически все «умным», то есть улучшает каждый аспект
    жизни с помощью сбора данных, алгоритмов искусственного интеллекта и сетей.
    Это может означать что-то такое же простое, как улучшение вашего холодильника и шкафов для обнаружения
    когда молоко и ваши любимые хлопья заканчиваются, а затем разместить заказ с вашими любимыми
    продуктами в интернет-магазине
    \item Возможности подключения — новые технологии для работы в сети и, в частности, Интернета вещей.
    сети, средние сети больше не привязаны исключительно к крупным провайдерам. сети
    может существовать в гораздо меньших и более дешевых масштабах, но при этом оставаться практичным. Интернет вещей создает
    эти небольшие сети между его системными устройствами.\cite{IoTAzure}
    \item Сенсоры. Интернет вещей теряет свои отличительные черты без сенсоров. Они действуют как определяющие инструменты
    которые превращают IoT из стандартной пассивной сети устройств в активную систему
    способный к интеграции в реальном мире.
    \item Активное взаимодействие — большая часть современного взаимодействия с подключенными технологиями происходит
    через пассивное взаимодействие. Интернет вещей представляет новую парадигму активного контента, продуктов,
    или участие в обслуживании.
    \item Маленькие устройства. Как и предполагалось, устройства стали меньше, дешевле и больше.
    мощный со временем. Интернет вещей использует специально созданные небольшие устройства для обеспечения точности,
    масштабируемость и универсальность.
\end{itemize}