\section{Безопасность IOT}
Каждое подключенное устройство создает возможности для \\злоумышленников. Эти уязвимости обширны, даже
для одного небольшого устройства. Представляемые риски включают передачу данных, доступ к устройству, неисправность
устройства, а также постоянно включенные/всегда подключенные устройства.\cite{DeisgOfIOT}
\begin{figure}[h!]
    \centering
    \includegraphics[scale=0.6]{security_issues.png}
    \caption{Возможные угрозы на разных уровнях представления}
    \label{fig:section7:security_issues}
\end{figure}
Основными проблемами в области безопасности остаются ограничения безопасности, связанные с производством недорогих устройств, и растущее количество устройств, создающее больше возможностей для атак.

Помимо затрат и повсеместного распространения устройств, IoT мешают другие проблемы безопасности:
\begin{itemize}
    \item Непредсказуемое поведение — огромное количество развернутых устройств и их длинный список
    обеспечивающие технологии означают, что их поведение в полевых условиях может быть непредсказуемым. Специфический
    система может быть хорошо спроектирована и находиться под административным контролем, но нет
    гарантии того, как он будет взаимодействовать с другими.

    \item Сходство устройств — устройства IoT довольно однородны. Они используют одно и то же соединение
    технологии и компоненты. Если одна система или устройство подвержены уязвимости, многие
    у многих такая же проблема.

    \item Проблемное развертывание. Одной из основных целей IoT остается размещение передовых
    сети и аналитика там, где раньше они не могли пройти. К сожалению, это создает
    проблема физической защиты устройств в этих странных или легкодоступных местах.
    
    \item Длительный срок службы устройства и просроченная поддержка. Одним из преимуществ устройств IoT является
    долговечность, однако этот долгий срок службы также означает, что они могут пережить поддержку своих устройств.
    Сравните это с традиционными системами, которые, как правило, имеют поддержку и обновления спустя долгое время.
    многие перестали их использовать. Осиротевшие устройства и брошенное ПО не имеют того же самого
    усиление безопасности других систем в связи с развитием технологий с течением времени.
    
    \item Отсутствие поддержки обновлений. Многие устройства IoT, как и многие мобильные и небольшие устройства, не поддерживаются.
    разработан, чтобы позволить обновления или любые модификации. Другие предлагают неудобные обновления,
    которые многие владельцы игнорируют или не замечают.
    
    \item Низкая прозрачность или отсутствие прозрачности. Многие устройства IoT не обеспечивают прозрачность в отношении
    к их функциональности. Пользователи не могут наблюдать или получать доступ к своим процессам, и им предоставляется
    предположим, как ведут себя устройства. У них нет контроля над нежелательными функциями или данными
    коллекция; кроме того, когда производитель обновляет устройство, это может принести больше
    нежелательные функции.
    
    \item Отсутствие предупреждений. Другой целью Интернета вещей остается обеспечение его невероятной функциональности без
    быть навязчивым. Это вводит проблему осведомленности пользователей. Пользователи не следят
    устройства или узнать, когда что-то пойдет не так. Нарушения безопасности могут сохраняться в течение длительного времени
    периоды без обнаружения.
\end{itemize}